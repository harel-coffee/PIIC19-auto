\documentclass[review]{elsarticle}

%%% Le da el formato final -- una vez aceptado
%\documentclass[final,authoryear,5p,times]{elsarticle}
%\documentclass[final,authoryear,5p,times,twocolumn]{elsarticle}

\usepackage{lineno,hyperref}
\modulolinenumbers[5]

\usepackage{amssymb}
\usepackage{amsmath}
\usepackage{amsfonts}
\usepackage{multirow}

\usepackage{algorithm} 
\usepackage{algpseudocode} 

%PO
\usepackage{color} % Use colors on text
\usepackage[normalem]{ulem} % Strike through text


\journal{Astronomy and Computing}

%% Elsevier bibliography styles
%%%%%%%%%%%%%%%%%%%%%%%
%% To change the style, put a % in front of the second line of the current style and
%% remove the % from the second line of the style you would like to use.
%%%%%%%%%%%%%%%%%%%%%%%
%% Harvard
\bibliographystyle{model2-names.bst}\biboptions{authoryear}
%%%%%%%%%%%%%%%%%%%%%%%


\begin{document}

\begin{frontmatter}

%\title{Variational Autoencoders with Scale for Light-curves: Learning a Deep Interpretable Representation and a Denoising Model Together!}
%\title{Learning Quality Deep Representations for Light Curves using Variational Autoencoders with Rescaling}
\title{On the Quality of Deep Representations for Unevenly-Sampled Light-Curves using Variational AutoEncoders}
%\tnotetext[mytitlenote]{Fully documented templates are available in the elsarticle package on \href{http://www.ctan.org/tex-archive/macros/latex/contrib/elsarticle}{CTAN}.}

% On the Quality of Deep Representations for Light-Curves using Variantional Autoencoders

\author[UTFSM_SJ]{Francisco Mena\corref{mycorrespondingauthor}}
\cortext[mycorrespondingauthor]{Corresponding author}
\ead{francisco.menat@usm.cl}

\author[UTFSM_CC]{Patricio Olivares}
%\ead[url]{www.elsevier.com}

\author[UTFSM_SJ]{Margarita Bugue\~no}
%\ead[url]{www.elsevier.com}

\author[UTFSM_SJ]{Gabriel Molina}
%\ead[url]{www.elsevier.com}

\author[UTFSM_CC]{Mauricio Araya}
%\ead[url]{www.elsevier.com}


\address[UTFSM_SJ]{Depto. Inform\'atica, Universidad T\'ecnica Federico Santa Mar\'ia, Santiago, Chile}
\address[UTFSM_CC]{Depto. Electr\'onica, Universidad T\'ecnica Federico Santa Mar\'ia, Valpara\'iso, Chile}


\begin{abstract} %motivation, problem, approach, result, concl
 Analyzing light curves nowadays usually involves processing large datasets. Therefore, finding a good representation for them is both, a key and a non trivial task. In this paper we show that variational (stochastic) autoencoder models (VRAE$_t$) can be applied to learn an effective, informative and robust deep representation of transit light curves.
In addition, we introduce S-VRAE$_t$, which stands for \textit{re-Scaling Variational Recurrent Auto Encoder}, a technique that embeds the re-scaling preprocessing of a time series into the learning model in order to use the scale information in the detection of transit exoplanets.
The objective is to achieve the most likely low dimensional output of unevenly-sampled time series that matches latent variables to reconstruct it.
For assessing our approach we use the largest transit dataset obtained by the Kepler mission in the past four years, and compare our results with similar techniques used for light curves. 
Our results show that the stochastic models have an improvement on the quality of representation with respect to their deterministic counterparts. Moreover, the S-VRAE$_t$ model is at the same time a deep denoising model, generating light curves similar to the Mandel Agol fit.
\end{abstract}

\begin{keyword}
Variational Autoencoder \sep  Transit Model \sep Light-Curve
\MSC[2010] 00-01\sep  99-00
\end{keyword}

\end{frontmatter}


\linenumbers

\section{Introduction}

There is not a clear unique way to detect and confirm exoplanets.
%This is because planets emit or reflect are very dim light magnitudes compared to their hosting stars and they are very near to the host star compared to the observation distance. 
%Therefore, it is a extremely difficult task, with only a couple of dozen exoplanets directly photographed while the majority of known exoplanets have been detected using indirect methods.
%The most successful indirect techniques study some variations on the host star product of its orbiting planet. For example,
Besides a couple of dozen exoplanets that have been directly photographed, analyzing the \textit{transit photometry}
%, used to analyze the light intensity (photometric observation)
and \textit{radial velocity} of the hosting star are the two more successful indirect observational methods for detecting exoplanets. However, the variations that an orbiting planet produces in its host could be very mild compared to other complex processes that a system may exhibit. 

Fortunately, technological advances in instrumentation and photometry have allowed experiments like the Kepler mission to have enough sensitivity for detecting a large number of exoplanets.
NASA has reported that more than 4000 exoplanet has been detected\footnote{http://exoplanets.nasa.gov} by grounded or spatial observatories, with $75\%$ of them using transit photometry. However, detecting an exoplanet from light curves is a comprehensive task that was primarily achieved through time-consuming manual processes. 
The increasing popularity of Machine Learning techniques and the large amount of data generated by observatories, allow the automatic learning of models that help astronomers to detect exoplanets by reducing effort and time. 
The common approach used to classify a light curve is to extract hand-crafted \textit{specialized features} from it and apply classic machine learning methods. For example, \citet{richards2011machine} present specialized features for variable star detection for the Catalina Real-Time Transient Survey (CRTS) and the Kepler Mission by \citet{donalek2013feature} using $k$-NN and Decision Tree models.
Similarly, \citet{hinners2017machine} use these features in order to predict the Kepler features using classic machine learning methods. The alternative is to use problem-learned representations, such as \citet{bugueno2018refining} introduce for the detection of exoplanet on the same Kepler dataset.

Deep Learning has been recently applied to face astronomy problems. In this domain, convolutional neural networks (CNNs) have been the most popular technique. For instance, \citet{shallue2018identifying} use a 1D convolutional neural network (CNN) model (\textit{Astronet}) to classify exoplanets on Kepler Mission with a global and local representation of the folded light curve. In the same vein, \citet{schanche2019machine} use a 1D CNN network to detect exoplanets among variable stars (4 classes) on WASP mission, and \citet{osborn2020rapid} use a 1D CNN network combined with scientific domain knowledge \citep{ansdell2018scientific} (\textit{Exonet}) on the recent TESS mission.
However, even though Machine Learning models are very helpful, most of the methods rely on hand-crafted features and external metadata to cope with the task. Therefore, a thorough and detailed analysis of the light curves only occurs once the metadata has been collected. 

This article focuses on detecting transients on light curves based only on the raw light curve data, in order to isolate the source of information. Cross-matched metadata is certainly important, but including it in the evaluation only hinders the effectiveness of the models used to tackle the time series complexity. The key idea of this article is to harness the power of Deep Learning for image processing, specifically 2D convolutional neural networks, and apply it to unevenly-sampled time series.
Concretely, we transform the light curves into a 2-channel image representation using and adapted version of Markov Transition Field (MTF) technique \citep{wang2015imaging}.
The first channel corresponds to the unevenly-sampled light curve measurements and the second channel represents the time information from the curve. The unevenly-sampled problem is addressed by defining semi-continuous transitions where each one represents measurements with a delta time below a maximum value.
This allows to extract temporal information from the raw light curve measurements. 

We assess our models using data from the Kepler mission, i.e. Kepler Objects of Interest (KOI\footnote{\texttt{http://archive.stsci.edu/search fields.php?mission=kepler\_koi}}) dataset. 
The results of our experiments show that we reach competitive performance and speed in the task of detecting exoplanets compared to using hand-crafted specialized features and similar deep learning approaches. 
%Our work achieves a macro-averaged F1-score of 77\%; 84\% in False Positive and 70\% in Confirmed classes. 
%which is obtained from a new curve which we call Sample Detection Curve (SDC) which counts the consecutive valid sample measures 

The article is organized as follows. Section 2 provides the background of the exoplanet detection problem with a discussion of the related work. After this, our proposal is introduced in Section 3. The experimental setting is presented in Section 4 with the corresponding results in Section 5.
An exploratory analysis of the method is discussed in Section 6 to finally conclude in Section 7.
%Finally, in Section 7 we conclude commenting on the extensions of our proposal.




%\section{Background}
%In astronomy exists different types of data that are processed, studied and analyzed, i.e. catalogs, time series and data cubes. 
%In the exoplanet field the data used are time series, particularly light curves. 
%A \textit{light curve} is defined as a graph of light intensity from a celestial object or region as a function of time, which can be obtained usually from photometric observation of a particular optical band. 
%Working with time series on astronomy has the challenge that they are quite uneven on their sampling, meaning that there are missing values from several timestamps or that measurements are not acquired uniformly.
%Some factors that produce these problems mostly affect ground observatories such as cloudy weather or Earth rotation.

%subsection*{Light Curves}
%When the light intensity of a star (luminosity magnitude) is measured, the resulting time series could varies its values as a result of the star composition or because of an orbiting planet. 
%The first case correspond to variable stars, which can be produced depending on the chemical composition of the star or because a celestial body (non necessarily planets) blocks the light. 
%By the other hand, if a planet is involved, it passes in front of the star and a fraction of the light is blocked (as an eclipse), this phenomenon is called \textbf{transit} and is an effective method to find exoplanets.

%existe otro?? SUPERNOVAS, MICROLENSING


\section{Transit Detection Methods} %  no me convence mucho, ya que las estrellas variables no son transitos.. algo mas general
%Light Curve Detection Methods
%Time series in Astronomy
\label{soa}

%In astronomy there exists different type of data that are processed, studied and analyzed, i.e. catalogs, time series and data cubes.
%In the exoplanet detection field the data type used is time series, particularly light curves.
A \textit{light curve} is a time series (function of time) with measurements of light intensity of a celestial object or region.
When one celestial body crosses in front of another astronomical object and blocks any fraction of its light, it is called a \textit{transit}. In this section, we briefly introduce model-based and self-learned representations used for detecting exoplanet transits and related fields.

\subsection{Model-based Representations}

The Mandel-Agol (M-A) simulation process \citep{mandel2002analytic}, models the transit of a spherical planet around a spherical star assuming an uniform light source. It does this by modeling the opacity observed on the light intensity $\ell c(t)$ according to the planet position. When the planet eclipses the star, the opacity is maximum ($\ell c<1$)). On the other hand, when the planet orbits without eclipsing the star, the opacity is minimum and uniform ($\ell c = 1$). When the planet is close to eclipse the star the intensity $\ell c(t)$ is modeled as a polynomial based on the limb darkening of the star. It requires to know the distance from the center of the planet to the center of the parent star, as well as the radius of each one of the bodies, the transit period, inclination and the limb darkening models (coefficients). There are a few Python libraries that have implemented this simulation, from which we selected \textbf{batman}\footnote{\texttt{github.com/lkreidberg/batman}}.

However, variable light curves are not caused only by transits, and several inherent and exogenous processes might be involved in a the variability on the observed intensity of a star. Therefore, several machine learning techniques have been used to classify variable stars, typically by manually extracting specialized features from the light curve and applying classic pattern recognition methods to them. 
For example, \citep{richards2011machine} presents a catalog of variable stars where 32 specialized features are extracted from light-curve through statistics such as kurtosis, skewness, standard deviation, and stetson, plus other features based on the period and frequency analysis of a Lomb-Scargle \citep{lomb1976least} fitted model. 
\citep{donalek2013feature} also worked on classifying variable stars from the Catalina Real-Time Transient Survey (CRTS) and the Kepler Mission, extracting similar features from the light curves. 
In \citep{nun2014supervised}, statistical descriptors are used as inputs for a Random Forest algorithm that can detect anomalous light curves based on probabilistic learning models. 
These outliers are removed from the training set used in variable stars classification. 

Specifically for transit detection we found the work of \citep{mccauliff2015automatic}, the so-called \textit{Autovetter}, which uses Random Forest over the features derived from the statistics pipeline on the Kepler mission for classifying candidate objects. 
Another approach is to represent the light curve as phase aligned sections called ``folds'' to tackle the irregular sampling problem in transits. This folded light curve centers the transit and stacks all the times on which occurs as shown in Figure~\ref{fig:lc_ex}. It usually get binned based on a window proportional to the period of the transit computed with a Lomb-Scargle \citep{lomb1976least} periodogram fit. For example, \citep{thompson2015machine} proposes a \textit{Locality Preserving Projection} (LPP) and \citep{armstrong2016transit} proposes using \textit{Self-Organization Map} (SOM) as a dimensionality reduction method of the folded light curve for finding transit shape objects on those extracted features. 

%\subsection{DL - Transit Detection}
Neural Networks and Deep Learning \citep{surveyDL2017} algorithms have become very popular on problems where feature extraction from data is non-trivial. These algorithms have been successfully used for transit detection on light curves in recent years. For example, \citep{shallue2018identifying} uses a 1D convolutional neural network (CNN) model \citep{krizhevsky2012imagenet} to classify exoplanets on the Kepler mission with a global and local representation of the folded light curve. 
\citep{pearson2018searching} used a similar approach to \citep{shallue2018identifying} for detecting transit shape objects trained on simulated data and evaluated using Kepler mission dataset.
\citep{schanche2019machine} also used a 1D CNN network to detect exoplanets among variable stars (4 classes) on WASP dataset. 

\subsection{Self-generated Representations}

While most of the representations focuses on using astrophysical knowledge to ease the classification of star variability or transits, only a few of them had tried different representation approaches without direct human intervention.
\citep{mackenzie2016clustering} used an unsupervised learning algorithm known as \textit{Affinity Propagation} 
%\citep{Frey2007affinityPropagation}
with a custom distance function to build a new representation from light curves and then use it on a linear SVM (\textit{Support Vector Machine}) classifier. The representation is based on the similarity between fragments of the light curves and cluster exemplars or centroids.

In \citep{mahabal2017deep} an image representation (i.e., grid) is obtained with variations of magnitude through time from unevenly-sampled light curves. This work used the variable star data from the Catalina-Real Time Transient Survey (CRTS) dataset and complete the classification task using 2-Dimensional Convolutional Neural Networks (CNN).
\citep{aguirre2019deep} also obtained the variations in magnitude from variable stars light curves and the delta time of each sample. These were used as different input channels to train a 1-Dimensional CNN classifier with shared-weights through novel data augmentation techniques.

\citep{naul2018recurrent} uses time as an additional channel, but through Recurrent Neural Network (RNN) models \citep{lipton2015critical}. Here it present a Recurrent Auto-Encoder (RAE) that learns an embedding of a light curve and then reconstructs it by setting the original times using RNN models on the encoder and decoder phase, that we named \textbf{RAE$_t$} (RAE \textit{plus} time information). \citep{naul2018recurrent} and \citep{tsang2019deep} show that learned representations are useful to classify variable stars, improving the results obtained using statistical features \citep{richards2011machine}. It also explores the use of the folded light curve representation improving even more the obtained results.

To the best of our knowledge, including delta times as an input to the models in unevenly-sampled times series was first explored by \citep{che2018recurrent}. This research presents a modification of a \textit{Gated Recurrent Unit} (GRU) model \citep{cho2014properties}, namely GRU-D, which uses a binary mask for missing values and the delta times as input channels. The objective is impute (fill) the missing values to improve the predictions on medical problems.


\subsection{Variational Auto-Encoders}
The Variational Auto-Encoder (VAE) is an stochastic Auto-Encoder (AE) learned in a probabilistic fashion, based on the variational lower bound or evidence lower bound (ELBO). 
The VAE framework \citep{kingma2013auto} is extended to work with uniform time series on the Variational Recurrent Auto-Encoder (VRAE) by \citep{fabius2014variational}. 
The motivation behind this is VAEs are deep generative models trained on an unsupervised scenario, learning latent variable representation as it trains. These variables are learned through the observed distribution, so it is built to adapt the variations on the behavior of the data. 
This is the main difference to \textit{vanilla} deterministic AEs that learn an invariable specific point for the input pattern \citep{vincent2010stacked}.

The use of the VRAE on different time series applications is associated to anomaly detection \citep{park2018multimodal,guo2018multidimensional,xu2018unsupervised}, with the objective of detecting outliers. It usually compares the reconstructed (or generated) input with the original values and set some threshold of tolerance to normal behavior. 
The sampled values from the latent distribution has smooth transitions \citep{kingma2013auto}, so the reconstructed data should reduce the bias of the specific patterns \citep{xu2018unsupervised}.
In addition, a VAE for generating transit-shape light curves as data augmentation technique is presented by \citep{woodward2019generating}.


%comentar sobre los denoising autoencoders? quizas el VAE logra lo mismo.. by introducing corrupted input with Gaussian noise,


\section{Extending VRAE to Uneven Samples}
\label{proposal}

Based on the effectiveness of deep stochastic models we propose two VRAE extensions that can properly handle dimensionality reduction of light curves with unevenly-sampled time series. The first model is a natural extension from VRAE that includes the time of the signal while the second model adds the scale information into the dimensionality reduction task.

\subsection{Problem Setup}

\begin{figure}[t!]
    \centering
    \includegraphics[width=0.95\textwidth, height=6cm]{imgs/LC_ex.png}
    \caption{Examples of light curves on Kepler mission. First column correspond to 4 years measurements with sampling rate of half an hour, while the second column correspond to the folded transit.}
    \label{fig:lc_ex}
\end{figure}

Consider a dataset $X = \{x^{(1)}, x^{(2)}, \ldots, x^{(N)}\}$, of $N$ input patterns $x$ distributed according to a unknown probability distribution $p(x)$. These inputs pattern are vectors of variable length, $x^{(i)}= (x^{(i)}_1, x^{(i)}_2, \ldots, x^{(i)}_{T_i})$, where $x^{(i)}_j \in \mathbb{R}$ represent the $j$-th observation of the time series $x^{(i)}$ of length $T_i$. Let ${t}^{(i)}_j$ be the timestamp when the $j$-th observation was obtained for data $i$. Furthermore, we define the time interval, or \textit{delta time}, for each observation: $\delta^{(i)}_j = t^{(i)}_j - t^{(i)}_{j-1} $, with $\delta^{(i)}_0 = 0 \ \forall i$.

This paper focuses on the transit-shape domain objects over the light curve $x^{(i)}$, in order to recognize patterns of exoplanets orbiting its host star. An example is shown on Figure \ref{fig:lc_ex}.

\subsection{VRAE for Unevenly-Sampled Time Series}

\begin{figure}[!t]
    \centering
    \includegraphics[width=0.9\textwidth]{imgs/BNaul_model.png}
    \caption{Diagram of the RAE$_t$ architecture for irregularly sampled time series data proposed by \citep{naul2018recurrent}. The sequence is processed by recurrent layers to produce a final fixed-length embedding with a single fully connected layer on last state. The decoder first repeats the fixed-length embedding $T_i$ times, and then appends the delta times ($\delta$, on figure $\Delta t$). The sampling times are passed to both the encoder and decoder; this is for determine the points at which the function should be evaluated.}
    \label{arch:bnaul}
\end{figure}

A Variational Auto-Encoder (VAE), like any Auto-Encoder architecture, is composed of a tangled encoder and decoder models trained on a unsupervised scenario\footnote{Unsupervised refers to the fact that no labels are used as inputs to the model.}. The encoder model $q_{\phi}(z|x)$, with parameters $\phi$, codifies the input pattern $x$ to a latent variable $z$, and the decoder model $p_\theta(x|z)$, with parameters $\theta$, reconstructs the input pattern from the codification $z$. The objective of the model is to maximize a (variational) \textit{lower bound} $\mathcal{L}(\theta,\phi; x^{(i)})$ of the log-likelihood $\ell(\theta,\phi; D)$ \citep{kingma2013auto}. For example, for an input pattern $x^{(i)}$ we have
\begin{align} \label{eq:basic_obj}
\ell \geq & \mathcal{L}(\theta,\phi; x^{(i)}) = \mathbb{E}_{q_{\phi}(z |x^{(i)})}\left[\log{p_{\theta}(x^{(i)},z)} -\log{q_{\phi}(z|x^{(i)})}  \right] \, \notag \\
    & \mathcal{L}(\theta,\phi; x^{(i)}) =  \mathbb{E}_{q_{\phi}(z |x^{(i)})}\left[ \log{p_{\theta}( x^{(i)} | z )}\right] - D_{\mbox{\tiny KL}}\left( q_{\phi}(z | x^{(i)}) || p_{\theta} (z)  \right) \, ,
\end{align}
where the first term of $\mathcal{L}$ is related to the expected reconstruction likelihood and the second enforces the consistency between the posterior obtained by the encoder $q_{\phi}(z|x)$ and some prior $p_\theta(z)$ (i.e., KL divergence). In the standard VAE, the distribution $q_{\phi}(z|x)$ is typically a normal $\mathcal{N}(\mu(x),\sigma(x))$ where $\mu(x), \sigma(x)$ are modeled by neural networks, while the common choices of $p_\theta(z)$ leads a KL divergence that can be integrated analytically. However, the first term of $\mathcal{L}$ needs to be approximated with the so-called \emph{re-parametrization trick}: $\hat{z} = \mu + \sigma \cdot \epsilon$, with an auxiliary noise variable $\epsilon\sim \mathcal{N}(0,1)$.

The shallow (fully connected) VAE for light curves proposed by \citep{woodward2019generating} could be extended to adapt RNN models into the encoder $q_{\phi}(z|x)$ and decoder $p_\theta(x|z)$ as a VRAE model.
However, to adapt the model for being used with unevenly-sampled light curves, we need the time intervals ($\delta$) as an extra input channel. 
To do this we follow the idea of the RAE$_t$ architecture proposed by \citep{naul2018recurrent} (shown in Figure \ref{arch:bnaul}), which adds the time information in both encoding and decoding sections of the autoencoder.
The proposed extension is called \textbf{VRAE$_t$} model (VRAE \textit{plus} time information). This model could be expressed by the encoder $q_{\phi}(z|x,\delta)$ and the decoder $p_\theta(x|z,\delta)$ which have the time information $\delta$ of every observation on the time series $x$ as an extra input signal.

%ventajas de VAE sobre AE (smooth light curve e independientes):
The motivation behind the variational extension is that the generative (stochastic) model learns a latent variable with smoother transitions, meaning that their latent space must be continuous \citep{kingma2013auto}.
%conectar con esto: (para entender que el modelo puede ser visto como un denoising) -- aprende los patrones de los datos mas que el ruido debido a su mayor capacidad de generalizacion que el AE normal ( buscar alguna referencia)
Since the model learns the distribution of the encoded variable, it becomes robust to input variations, similarly to a denoising Auto-Encoder \citep{vincent2010stacked}. Indeed, the learned distribution must have more likely regions or confidence intervals where the input data should be.
%OBJ de denoising AE: a good representation is one that can beobtained robustly from a corrupted input and that will be useful for recovering the correspondingclean inpu


\subsection{VRAE with Embedded Re-scaling}

Currently, deep learning methods need a standardized version of the input representation $x^{(i)}$ that retains the original distribution but re-scaled to more tractable magnitudes. This is used for properly training neural network models, based on the generalization principle that magnitudes of weights and activation functions must be somehow bounded \citep{bishop1995neural,montavon2012neural}. 
Furthermore, \citep{ioffe2015batch} recommend that a normalization step is added after each layer to get a more stable training. 
On a time series, this transformation is usually based on its own magnitude behavior or \textbf{scale}.  %"definicion" de cómo se trata el concepto
For example $x'^{(i)} = (x^{(i)} - \mathrm{min}(x^{(i)}))/(\mathrm{max}(x^{(i)}) - \mathrm{min}(x^{(i)}))$ change the range magnitudes to $[0,1]$, or $x'^{(i)} = (x^{(i)}-\mathrm{mean}(x^{(i)}))/ \mathrm{std}(x^{(i)})$ could lead $x'^{(i)} \sim \mathcal{N}( 0,1)$ if $x^{(i)}$ is normal distributed. These transformations are necessary for the model to detect pattern behaviors instead of magnitudes behaviors. The problem is that the scale information is lost on the process. 
Here we propose to add the scale $s^{(i)}$ of the time series $x^{(i)}$ as an additional input to the VRAE$_t$ model, but still use the re-scaled version of the data for achieving bounded weights and activations. To the best of our knowledge, this is the first approach to do this as a end-to-end architecture.

Assuming that a time series $x^{(i)}$ is de-trendend( with zero mean), we can define its own scale as $s^{(i)} = std(x^{(i)})$. %, as each one is de-trended with zero mean. 
We consider that the scale $s^{(i)}$ is another input pattern of the autoencoder model VRAE$_t$ that needs to be reconstructed too. With this, the objective of the VRAE$_t$ with \textit{re-Scaling} or \textbf{S-VRAE$_t$}  is to reconstruct the time series on the original (raw) scale magnitude.

\subsubsection{S-VRAE$_t$ Architecture}
The outline of the main components of the S-VRAE$_t$ model, which clarify the differences with respect to VRAE$_t$ are summarized here:
%OLDVERSION: on REspaldo/old_point

\begin{enumerate}
    \item \textbf{Re-scale Data} \ \ The first layer of the encoder $q_{\phi}(\cdot)$ re-scales the data by dividing on the original scale. This step is performed in order to use the standardized version of the data, as the literature recommends.
    \item \textbf{Encode}. The encoder $q_{\phi}(\cdot)$ adds the scale as an input pattern to the coding task by $q_{\phi}(z^{(i)}|x^{(i)},\delta^{(i)}, s^{(i)})=\mathcal{N}(\mu^{(i)}, \sigma^{(i)})$ in order to extract the information lay up on the scale. 
    %by the \emph{re-parametrization trick}
    \item \textbf{Sample} \ \ The sampled latent variable is given by:  $\hat{z}^{(i)} \sim \mathcal{N}(\mu^{(i)} , \sigma^{(i)})$.
    \item \textbf{Reconstruct} \ \ The decoder $p_\theta(x^{(i)}|z^{(i)},\delta^{(i)})$ adds the scale to the reconstruction task in order to get the original raw scale by $p_\theta(s^{(i)}|z^{(i)})$, 
    \item \textbf{Re-scale Reconstruction} \ \ Last layer of the decoder re-scale the data, by returning the reconstructed scale (multiply by it). This final step is performed in order to get a reconstructed time series on the original raw scale representation.
\end{enumerate}
Besides, inspired by \citep{ioffe2015batch}, a Normalization Logarithm ($Norm_L$) layer is introduced to handle high variable magnitude values. Defining an input tensor $a$, the forward pass is given by:
\begin{equation}
     Norm_L(a) = \frac{ \log{a} - mean(\log{a}) }{ std(\log{a})}
\end{equation}
Also, another layer is introduced to reverse this transformation: $RevNorm_L(a) = Norm_L^{-1}(a) = exp(a \cdot std(\log{a}) + mean(\log{a}))$. The \textit{mean} and standard deviation (\textit{std}) were previously computed over the whole the dataset.

\begin{figure}[t!]
\begin{minipage}[t]{0.48\textwidth}
\centering
\begin{algorithm}[H] 
\caption{Forward pass VRAE$_t$}
\label{alg:vrae}
\hspace*{\algorithmicindent} \textbf{Input}: $x_s^{(i)}$ - scaled time series\\ %measurements \\
\hspace*{\algorithmicindent} \hspace{0.18\textwidth} $\delta^{(\ell)}$ - time series delta time  \\
\hspace*{\algorithmicindent} \textbf{Output}:  $\hat{x}_s^{(i)}$ - scaled reconstructed time series
\begin{algorithmic}[1]
\State //Encode to distribution:% parameters: 
\State  $\mu^{(i)} \gets f_{\phi}^1\left( E^1(x_s^{(i)}, \delta^{(i)}) \right)$
\State $\sigma^{(i)} \gets f_{\phi}^2\left( E^1(x_s^{(i)}, \delta^{(i)}) \right)$
\State $\epsilon\sim \mathcal{N}(0,1)$ //auxiliary noise
\State $\hat{z}^{(i)} \gets \mu^{(i)} + \sigma^{(i)} \cdot \epsilon$
\State //Decode or Reconstruct:
\State $\hat{x}_s^{(i)} \gets g_{\theta}^1\left( z^{(i)}, \delta^{(i)} \right)$
\end{algorithmic} 
\end{algorithm}
\end{minipage}
\hfill
\begin{minipage}[t]{0.48\textwidth}
\centering
\begin{algorithm}[H] 
\caption{Forward pass S-VRAE$_t$}
\label{alg:s-vrae}
\hspace*{\algorithmicindent} \textbf{Input}: $x^{(i)}$ - time series\\ % measurements \\
\hspace*{\algorithmicindent} \hspace{0.18\textwidth} $\delta^{(\ell)}$ - time series delta time \\
\hspace*{\algorithmicindent} \textbf{Output}:  $\hat{x}^{(i)}$ - reconstructed time series
\begin{algorithmic}[1]
\State $s^{(i)} \gets std(x^{(i)})$
\State  $x_s^{(i)} \gets \frac{x^{(i)}}{s^{(i)}}$ //Step-1
\State //Encode to distribution: Step-2 % parameters: Step-2
\State  $\mu^{(i)} \gets f_{\phi}^1\left( E^1(x_s^{(i)}, \delta^{(i)}), E^2(s^{(i)}) \right)$
\State $\sigma^{(i)} \gets f_{\phi}^2\left( E^1(x_s^{(i)}, \delta^{(i)}), E^2(s^{(i)}) \right)$
\State $\epsilon\sim \mathcal{N}(0,1)$ //auxiliary noise
\State $\hat{z}^{(i)} \gets \mu^{(i)} + \sigma^{(i)} \cdot \epsilon$ //Step-3
\State //Decode or Reconstruct: Step-4
\State $\hat{x}_s^{(i)} \gets g_{\theta}^1\left( z^{(i)}, \delta^{(i)} \right)$
\State $\hat{s}^{(i)} \gets g_{\theta}^2\left( z^{(i)} \right)$
\State $\hat{x}^{(i)} \gets \hat{x}_s^{(i)} \cdot \hat{s}^{(i)}$ //Step-5
\end{algorithmic} 
\end{algorithm}
\end{minipage}
\end{figure}
A pseudo-code of the S-VRAE$_t$ forward pass is presented on Algorithm \ref{alg:s-vrae}, comparing against the VRAE$_t$ on Algorithm \ref{alg:vrae}. Here it can be seen that the main differences are the re-scaling process inside the model and the additional input pattern to use. The first thing to formalize is that $g_w(\cdot)$ and $f_w(\cdot)$ are non-linear functions (i.e., deep learning models) parameterized by $w$. The $E(\cdot)$, which stands for \textit{embedding} function, corresponds to the first layers of a deep learning model. Inside $f_{\phi}(\cdot)$, the $E^1(\cdot)$ is a RNN model and $E^2(\cdot)$ a MLP or FF (\textit{Feed Forward}) model with first layer $Norm_L(\cdot)$. Here, the $E^1$ model codifies the information from the unevenly-sampled standardized time series, while $E^2$ codifies the scale. 
On the decoder phase, the $g^1(\cdot)$ is similar to a mirror model of $E^1(\cdot)$, while reversing what $E^2(\cdot)$ does, the last layer of $g^2(\cdot)$ is $RevNorm_L(\cdot)$.
Here $g^1$ reconstruct the standardized version of the time series and $g^2$ reconstruct the scale. 


\begin{figure}[!t]
    \centering
    \includegraphics[width=\textwidth]{imgs/svrae_model.pdf}
    \caption{Diagram of the S-VRAE$_t$ architecture for irregularly sampled time series. Firstly, the raw time series input is re-scaled and pass to the RNN encoder block together with the delta times. In parallel the scale input is encoded too. A concatenation is perform on both learned embedding values to obtain the Normal distribution parameters of the latent variable. After a sample is performed over this distribution, the value is repeated and concatenated with the delta times in order to reconstruct the scaled time series and the original scale. Finally the raw scale is returned to the time series.}
    \label{arch:svrae_ill}
\end{figure}
The  architecture of the proposed \textbf{S-VRAE$_t$} is illustrated on Figure \ref{arch:svrae_ill}. Please note that the Encode and Reconstruct blocks are the same of the RAE$_t$ on Figure~\ref{arch:bnaul} but without the scale transformations. 
%BENEFICIOS/Ventajas/CONTRIBUCIONES
In summary, S-VRAE$_t$ learns a coded \textit{deep representation} of the unevenly-sampled raw time series in an unsupervised way. The advantage is that those features are optimized for the specific input behavior (rather than classification) and have less bias than human-crafted counterparts. The evidence for this claim can be found in Section \ref{res}.

%denoising -- era la motivacion original
In any VAE model, the objective of learning the distribution of the latent variable is to obtain the more likely reconstructed input pattern. In our S-VRAE$_t$ model the time series gets reconstructed on the original scale, $\hat{x}^{(i)}$, so we can consider it as a smoothing or denoising model on a data-dependent way (based on the data behavior).
As the model processes the scale inside it, allows to remove the noise inside the scale.
For example, consider a decomposition of the time series measurements based on $x^{(i)} =  x_p^{(i)} + n^{(i)}$, with $n^{(i)}$ the intrinsic noise of every measurement on $x_p^{(i)}$, and $x_p^{(i)}$ the time series measurements with isolated or perfect environment, as the one modeled by \citep{mandel2002analytic}. 
On this case, the denoised scale would be $s_p^{(i)} = std(x_p^{(i)})$ and could only be learned by S-VRAE$_t$ in order to re-scale the reconstructed denoised time series $\hat{x}_p^{(i)} = \hat{x}_{p,s}^{(i)}\cdot s_p^{(i)}$. 
Any other model with the re-scaling process fixed would use a noised scale, for example, $s^{(i)} = std(x^{(i)}) = std(x_p^{(i)} + n^{(i)})$.
This shows the capacity of S-VRAE$_t$ to denoise raw time series.
%based on the previous decomposition we can express the square of the scale (or the variance) by 
%$(s^{(i)})^2 = std(x^{(i)})^2 = var(x_p^{(i)} + n^{(i)})= var(x_p^{(i)}) +  var(n^{(i)}) + 2 cov(x_p^{(i)},n^{(i)}) $, with $cov$ the covariance.

%On the other hand, models like RAE$_t$ and VRAE$_t$ have to process (i.e., encode and reconstruct) the standardized version of the data $x_s^{(i)}$. Then, from outside the model the original scale could be returned, with the real value ($s^{(i)}$), this is $\hat{x}^{(i)} = \hat{x}_s^{(i)}\cdot s^{(i)}$. The problem is that there could be is noise inside the real scale, $s^{(i)} = std(x^{(i)}) \approx std(\check{x}^{(i)} + n^{(i)}) = std(\check{x}^{(i)}) +  std(n^{(i)}) $, with $n^{(i)}$ the intrinsic noise of every measurement on the time series $\check{x}^{(i)}$, and $\check{x}^{(i)}$ the time series measurement with isolated or perfect environment, as the one simulated by M-A model \citep{mandel2002analytic}.


%However, smoothing or denoising a light curve is not straightforward for transits, because the transits can be seen as isolated behaviors or anomalies (i.e., threshold events) as Figure \ref{fig:representation_ex} shows. Therefore, a \textit{vanilla} denoising model will remove transits by considering them anomaly deviations of the ``normal'' star magnitude.


%Optimization:
\subsection{Loss Function}
In this section we describe the optimization objectives of the Variational Auto-Encoder models (VRAE$_t$ and S-VRAE$_t$) for time series domain.

\textbf{Reconstruction loss} \ \ 
We use a modified version of the mean squared error (\textit{MSE}) loss function for time series, named \textit{weighted} or \textit{re-scaled} \textit{MSE}, given by
\begin{equation}
    SMSE(X, \hat{X}) = \frac{1}{N} \sum_{i=1}^N w^{(i)} \cdot \frac{1}{T_i} \sum_{j=1}^{T_i} \left( x_j^{(i)} - \hat{x}_j^{(i)}  \right)^2 \, ,
\end{equation}
where weight $w^{(i)}$ is associated to every input pattern (as a \textit{sample weight} \citep{freund1995desicion}) and is defined as the inverse to variance $w^{(i)} = 1/var(x^{(i)})$. The weight value is derived from $w_i = \left(1/s^{(i)}\right)^2$, with the idea of remove the intrinsic scale (standard deviation) of the time series $x^{(i)}$, and so every input pattern has the same impact on the objective. In order to perform a proper reconstruction of the input patterns, the SMSE would be the loss function of the deterministic counterpart of VRAE$_t$, this is, the RAE$_t$ model.

\textbf{Variational loss} \ \ The two factors to optimize on the variational lower bound (Equation \ref{eq:basic_obj}) could be expressed by: i) a reconstruction factor through the \textit{SMSE} described above, and ii) the closed solution of a \textit{KL} divergence with Normal priors \citep{kingma2013auto}.
%The Variational lower bound of Equation \ref{eq:basic_obj}, has two factors to optimize. The first factor (reconstruction) is expressed through the \textit{SMSE} described above, while the \textit{KL} divergence is expressed as the closed form solution with Normal priors distribution presented by \citep{kingma2013auto}. 
Following \citep{higgins2017beta}, we combine these two factors using a regularization parameter $\beta$ obtaining the loss function of the VRAE$_t$ model:
\begin{equation}
    V(X, \hat{X}) = SMSE(X, \hat{X}) + \beta \cdot KL(Z) \, .
\end{equation}
Our earlier experiments on validation set led us to set the $\beta$ parameter to a small value of $10^{-3}$. This mean that the priority is set on the reconstruction task but with a variational component.


\textbf{Variational loss with re-Scaling} \ \ As our proposed S-VRAE$_t$ adds the scale to the reconstruction task, we need to specify an additional loss for this. We used the mean squared logarithm error (\textit{MSLE}) as loss function, given by %to scale reconstruction:
\begin{equation}
MSLE(S, \hat{S}) = \frac{1}{N} \sum_{i=1}^N \left( \log{s^{(i)}} - \log{\hat{s}^{(i)}}  \right)^2 \, .
\end{equation}
Then, the learning objective of S-VRAE$_t$ model is given by:
\begin{equation}
    SV \left( (X,S) , (\hat{X},\hat{S}) \right)=  V(X, \hat{X}) + \alpha \cdot MSLE(S, \hat{S}) \, ,
\end{equation}
where the $\alpha$ parameter was set to $10^{-1}/var(\log{S})$. The same motivation used for $\beta$ led to set the $10^{-1}$ value, giving firstly relevance to the reconstruction loss \textit{SMSE} inside \textit{V}. While the $var(\log{S})$ value is to re-scale the magnitude of the \textit{MSLE} loss (by removing the standard deviation) to the same proportions of \textit{SMSE} loss.





\section{Experimental Setup}
\label{exp}

\subsection{Dataset}
%no hablar mucho de etiquetas puesto que es solo para evaluacion... es una tecncia no supervisada.
%Currently, the dataset with the most studied exoplanets is from the NASA based on the effectiveness of Kepler mission\footnote{Kepler measured the light variation of thousand of distant stars in search of periodic planetary transit within our galaxy neighborhood.}. Particularly,

Our work uses the \textit{Kepler Objects of Interest} (KOI) dataset\footnote{http://archive.stsci.edu/search\_fields.php?mission=kepler\_koi} provided by MAST (\textit{Mikulski Archive for Space Telescopes}) \citep{akeson2013nasa}. It is composed by 8054 records where every Kepler Object of Interest (KOI) is a region of interest in the sky that shows a periodic behavior based on thresholding events. These objects are categorized according to Nasa Exoplanet Science Institute\footnote{http://nexsci.caltech.edu/}, as \textit{Confirmed}, claimed exoplanets through extensive scientific analysis and follow ups; \textit{False Positive}, initially selected as candidate exoplanets but additional evidence show they are not; or \textit{Candidate}, those that are still under study (unlabeled data).
Multiple KOIs are obtained from each host star, and each object contains raw measurements with timestamps, including the instrumental error associated to each measure. The sampling rate is 0.0204 BJD on average (i.e., half an hour), but there is a 22.98\% of missing values per light curve on average. Each light curve has approximately 55000 effective measurements.

The archive also provides a set of metadata values, some of them (usually related to the star) are cross-matched from other catalogues. For comparing our self-generated features with model-based ones, we have selected only those that can be directly obtained from the light curves by following the Mandel-Agol model \citep{mandel2002analytic}.
%\emph{period}, \emph{First Transit Time}

%\subsubsection*{\textbf{Metadata}}
%From all the high-level features (metadata) that was available, we have selected the features that can be extracted only from the light curve, without cross-matching from other cataloges.
%\begin{itemize}
%\item \emph{Period}: the interval between consecutive planetary transits in days. Calculated as one of the best-fit parameter on a Mandel-Agol model \citep{mandel2002analytic}.
%\item \emph{First Transit Time}: the time when the exoplanet pass in front of the host star in BJD. Calculated as one of the best-fit parameter on a Mandel-Agol model \citep{mandel2002analytic}.
%\item \emph{Inclination}: the angle between the plane of the sky (perpendicular to the line of sight) and the orbital plane of the object (90 degrees is a orbit in the line of sight). Calculated as one of the best-fit parameter on a Mandel-Agol model \citep{mandel2002analytic}.
%\item \emph{Planet Radius} (over Stellar Radius): the inferred radius of the Object of Interest. Based on a best-fit parameter on a Mandel-Agol model \citep{mandel2002analytic}. Calculated as one of the best-fit parameter on a Mandel-Agol model \citep{mandel2002analytic}.
%\item \emph{Semi-major Axis} (over Stellar Radius): orbital radius based on the axis of an elliptic orbit. Calculated as one of the best-fit parameter on a Mandel-Agol model \citep{mandel2002analytic}.
%\item \emph{Limb Darkening Coefficients}: models the light variation (darkening) on the edges of the star, two coefficient (linear and quadratic). To be sure, Kepler used cross-matching.

%\item \emph{Impact Parameter}: distance between the object and the sight axis to the star. Calculated based on the orbit radius and the inclination.
%\item \emph{Transit Duration}: time between first contact and last contact in days (eclipse duration). Calculated based on the transit period, planet radius and orbit radius.
%\item \emph{Fitted stellar density}: calculated based on a small planet approximation with the planet radius, transit period and duration.
%\item \emph{Planet Teq}: the expected temperature of equilibrium on the surface of the candidate planet in Kelvin. Calculated through the stellar temperature and the planet's radiation.
%\end{itemize}


%\subsection{Dataset Pre-processing}
We use Kepler's detrend pipeline \citep{fanelli2011kepler} to obtain a standard light curve for our experiments, i.e. that express the changes respect to a local window behavior. It consist on applying
%\begin{enumerate}
%    \item Apply 
a polynomial fit, Sav-Gol filter \citep{savitzky1964smoothing}, of 2 degrees with a window of 151, and then subtract it to the light curve. Then it
   % \item Apply  and 
subtracts a moving median filter, with a window of 25.
Finally, we removes
%    \item Remove 
positive outliers that are high than $5\sigma$ and removes negative outliers lower than $40 \sigma$. 
%With $\sigma$ the standard deviation of the light curve.
%\end{enumerate}


\subsection{Data Representation}
\begin{figure}[!t]
    \centering
    \begin{tabular}{c}
        Example 1 \\
         \includegraphics[width=0.95\textwidth, height=3cm]{imgs/conf_ex.png}  \\
        Example 2 \\ 
         \includegraphics[width=0.95\textwidth, height=3cm]{imgs/narrow_ex.png} \\
         Example 3 \\ 
         \includegraphics[width=0.95\textwidth, height=3cm]{imgs/noisy_ex.png} 
    \end{tabular}
    \caption{Examples on the different representation that can be obtained from a periodic light curve by knowing the period $P$. The images on every example is the raw light curve, the folded and folded-global (setting the bins to $T=1000$) on the columns respectively. The third column is the representation used on this work.}
    \label{fig:representation_ex}
\end{figure}

For the light curve measurement $x^{(i)}$, we use the \textbf{folded-global} representation proposed by \citep{shallue2018identifying}. First it produce a vector $f^{(i)}$ with the same number of points by folding the raw light curve on the period ($P$), with the event centered (fold step). Then, a windowed median is applied every $P/T$ times over the folded light curve $f^{(i)}$ (global step). This produces a vector $\mathbf{x}^{(i)}$ with $T$ points, so each light curve has the same length with the width interval depending on the period $P$.
%This representation was original used to classify \textit{Thresholding Crossing Event} (TCE)\footnote{This is detect Candidates from False Positive objects} with a good performance. The process is described here:
%\begin{enumerate}
%    \item \textit{Fold}: Produce a vector $f^{(i)}$ with the same number of points by folding the raw light curve on the period ($P$), with the event centered (need the \textit{first time transit} measure).
%    \item \textit{Global}: A window median is applied every $P/T$ times, on the folded light curve $f^{(i)}$. Producing a vector $\mathbf{x}^{(i)}$ with $T$ points, so each light curve has the same length with the width interval depending on the period $P$.
%\end{enumerate}
The parameter $T$ controls the trade-off between a detailed representation and having enough points on each window for the median to be meaningful.

Figure \ref{fig:representation_ex} shows examples of folded and folded-global representations of light curves.
The second example illustrates a general disadvantage of this method, where long-period KOIs may end up with very narrow transits that fall entirely within a small number of bins. On the other hand, the third example shows how the folded-global representation helps to get a more cleaned version of the light curve when the planets are small. 

As explained in Section~\ref{proposal}, the delta times ($\delta^{(i)}$) of the folded-global representation are considered as an additional input for the learning models. 
Also, please recall that the light curve measurements $\mathbf{x}^{(i)}$ are previously normalized through $\mathbf{x}'^{(i)} = \mathbf{x}^{(i)}/std(\mathbf{x}^{(i)})$ for VRAE$_t$,%, as pre-process let the light curves centered ($mean(\mathbf{x}^{(i)}) = 0 \ \forall i$). 
while the raw light curve measurements $\mathbf{x}^{(i)}$ are used for the proposed S-VRAE$_t$ model.


\subsection{Data Selection and Augmentation} %and Further Processes
%We set a mask over all the objects on Kepler mission (8054 records) to filter out light curves without a transit behavior, obtaining 4317 objects to train our models. 
%Concretely, we remove objects that either (i) are classified as ``secondary transit" or ``not transit", (ii) have a \textit{transit score} lower than $0.55$, or (iii) have a Mandel-Agol residual higher than $1$.
%The candidates objects that are not transit could be because the transit was from another object (non-planetary) in the background or because it was a binary system, i.e. two stars orbiting around their barycenter.

We set a mask over all the objects on Kepler mission (8054 records) to filter out light curves without a transit behavior, obtaining 4317 objects to train our models. The process is described below:
\begin{itemize}
    \item Check for Kepler \textit{flags} (metadata) and remove objects with ``secondary transit" or ``not transit" flags.
    \item Check for \textit{transit score} (Kepler metadata) and remove objects with value lower than $0.55$.
    \item Perform a Mandel-Agol fit and check the residual, remove the object with \textit{SMSE} higher tan $1$.
\end{itemize}
%Between the reason to set the flags on the Kepler pipeline we found that the observation did not match with the star position on study, for instance because the transit was from another object (non-planetary) in the background (``not transit" flag).
%Another possibility is that the deep of the even transit was statistically different to the deep of the odd transits, showing a binary system, i.e two stars orbiting among them (``secondary transit" flag).
The candidates objects that are not transit could be because the transit was from another object (non-planetary) in the background (``not transit" flags) or because it was a binary system, i.e. two stars orbiting around their barycenter, so it has a second statistically different transit (``secondary transit" flag).

Also, as a data augmentation step \citep{tiensuu2019detecting}, we double this dataset by mirroring each folded light curve. This represents that the same object, with the same properties, orbits the star in the opposite direction. 


\subsection{Model Implementation}
Following the RAE$_t$ model \citep{naul2018recurrent}, our VRAE and VRAE$_t$ models implement GRU over LSTM on the recurrent layers as it presents roughly the same performance \citep{chung2014empirical}, but has fewer parameters and needs to store less information per time series (i.e., it is faster). The encoder and decoder recurrent section of the models (i.e., $E^1$ and $g^1$ on Algorithms \ref{alg:vrae} and \ref{alg:s-vrae}) are a stack of 2 bidirectional RNN layers of 64 units, in order to increase representation complexity as \citep{naul2018recurrent} present. We implement our model using the Keras library\footnote{\texttt{https://keras.io}}.   %codigo


\subsection{Model Assessment}

%test seT???

\subsubsection*{Encoder-Decoder Evaluation} %AE/VAE

For assessing the \textit{reconstruction} quality, we compare the estimated input pattern $\hat{x}$ and real value $x$, using the Root Mean Squared Error (\textbf{RMSE}) and Mean Absolute Error (\textbf{MAE}) as performance metrics. For evaluating the \textit{denoising} effect, we use the estimated input pattern $\hat{x}$ to measure the autocorrelation (\textbf{Autocorr}) and the Mean of the Differences (\textbf{Diff-M}) between consecutive values. A high autocorrelation and low mean difference, is associated to a smoother time series. For measuring the structure left within the \textit{residual noise} (difference between estimated and real input pattern), we use
%Based on the residual of model outputs, i.e. difference between estimated input pattern $\hat{x}$ and real value $x$. A 
an information theory score called the Spectral Entropy (\textbf{Spectral-H}) \citep{inouye1991quantification}. A high entropy value means a less structured residual in terms of signal frequencies.

Smoothing a time series is a classical signal processing task, but unfortunately this is not straightforward for transits, because transits are (statistically speaking) isolated behaviors or anomalies (i.e., threshold events) as Figure \ref{fig:representation_ex} shows. Therefore, a \textit{vanilla} denoising model will remove transits by considering them anomaly deviations of the ``normal'' star magnitude.
Based on this, we have selected the following baselines to compare our models: the Butterworth passband filter \citep{chandrakar2013survey}, a moving average filter with different window size, a Mandel-Agol simulation based on the object metadata and the Recurrent Auto-Encoder plus time (RAE$_t$) by \citep{naul2018recurrent} .

\subsubsection*{Encoder evaluation} %deep representation
For evaluating the deep representation quality learned by the encoder sub-architecture, we analyzed how it performs in a classification task, and how orthogonal the generated features are.
The \textit{classification} was made using a Feed Forward (FF) network built over the representation with 128 units and \textit{relu} activation, ending with a \textit{sigmoid} classification layer (1-unit) as output. The model is trained using the dataset labels (2281 exoplanets and 3976 non-exoplanets). A F1 macro criterion (\textbf{F1-Ma}) is used to assess the results. 

For analyzing the \textit{features dependence} a Pearson correlation between all the features on the representation was measured to find linear dependence. We report the average over all the values (\textbf{Pcorr}) and the average over the absolutes values (\textbf{Pcorr-A}).  
The Mutual Information between the continuous features is measured as well (\textbf{MI}), including a normalized version (value between $[0,1]$) that is obtained by dividing on the entropy of every feature (\textbf{N-MI}). The calculation of Mutual Information is a discrete approximation of the real continuous space calculation, based on the $k$-neighbors implementation \citep{ross2014mutual}.
    %\item \textit{Clustering}: ??
    %\item \textit{Interpretability}: Pendiente

Here we compare with the high-level specialized features included in Kepler's metadata, the features learned by a Recurrent Auto-Encoder plus time (RAE$_t$) \citep{naul2018recurrent}, and the PCA features extracted over the frequency domain (spectrum) representation of the time series (F+PCA) \citep{bugueno2018refining}. 


%Decoder using:
%para generar curvas y agregar interpretabilidad

\section{Results}
The results of our experiments were obtained thanks to the ChiVO\footnote{\texttt{https://chivo.cl/}} (Chilean Virtual Observatory) datacenter \citep{solar2015chilean} by using an Intel Xeon CPU E5-2680 2.50GHz with 12 cores and 64 GB of RAM.

\subsection{Number of States Impact}
\begin{table}[!t]
    \caption{Macro averaged F1-score on validation set for different number of settings. $n_{up}$ stands for the number of positive states while $n_{down}$ corresponds to negative states defined for the MTF representation. $*$ symbol shows the best result.}
    \label{tab:exp_states}
    \centering
    \begin{tabular}{c|ccccc}
        \diaghead(-2,1){\hspace{1.7cm}}{$n_{up}$}{$n_{down}$}& %do not change
            8 & 16 & 32 & 64 \\ \hline
        4  & 71.48 & 72.66 & 72.97 & 71.81 \\ 
        8  & 72.14 & 73.35 & 74.43 & 72.00 \\ 
        16 & 72.84 & 74.69 & 75.75$^*$ & 74.13  \\ 
        32 & 73.84 & 75.03 & 75.40 & 75.53 \\  
        64 & 72.75 & 73.47 & 74.03 & 74.95  \\ 
        \hline
        %128 & $\times$ & $\times$ & $\times$ & $\times$ & $\times$ %& Si
    \end{tabular}
\end{table}
Table \ref{tab:exp_states} presents the experimental impact of the number of states in the classification task. 
The best \textit{F1-score} was obtained by using $n_{up} = 16$ and $n_{down}=32$, which is a 2-channel image of $48\times 48$ size.
Note that as the number of states increases (both $n_{up}$ and $n_{down}$), the performance improves, this is because the model input becomes a fine-grained representation (more detailed image).  
However, for any $n_{up}$ up to 16 states, macro averaged \textit{F1-score} reaches a maximum with $n_{down}=32$, then it decays. %(with $n_{down}=64$). 
The experimental results show that the number of transitions does not need a more fine-grained representation ($n_{down} \geq 64$) because the transit information turns fuzzy, so each pixel of the image does not represent aggregated information. In that sense, the MTF image results in a very sparse matrix, since there is a lot of states but only a few transitions per state. 

By using $n_{up}=32$ or $n_{up}=64$, when more $n_{down}$ states are used a higher \textit{F1-score} is reached.
This shows that the positive-negative symmetric states, i.e. $n_{up}=n_{down}$, are not always the best setting.
Most of the cases reach better results when a higher number of states on the negative values is set.
Indeed, given an $n_{up}$, the number of negative states ($n_{down}$) should be higher or equal to $n_{up}$. 
This makes sense if we analyze the problem context since the negative transitions (light blocking) are more important than the positive ones (noise and stellar properties).

\subsection{Method Comparison}
Using the best setting on validation set, $48\times 48$ images ($n_{up} = 16 $ and $n_{down} = 32$), we obtain the final results on the test set against the compared methods.

\begin{table}[!t]
\caption{Classification performance using different methods and feature extraction techniques. The \textit{F1-score} per class (C: \textit{Confirmed}, FP: \textit{False Positive}) and the macro averaged \textit{F1-score} are presented. $\star$ The feets results reported corresponds to a sample of 2500 objects (2 months execution).}
\label{tab:class_results}
\begin{tabular}{c|c|cc|c} \hline
\textbf{Method} & \textbf{Input shape} & \textbf{FP class}& \textbf{C class} & \textbf{Macro avg.} \\ \hline 
\multicolumn{5}{c}{\textit{Specialized hand-crafted features + Classic Learning Methods}} \\ \hline
Metadata  & $10$                 & $90.13$ & $83.85$  & $87.00$ \\ %\hline
feets$^{\star}$     & $57$                 & $84.57$ & $31.19$  & $57.88$    \\ \hline
\multicolumn{5}{c}{\textit{Feature extraction + Classic Learning Methods}} \\ \hline
F-PCA  & $32$                 & $80.03$ & $58.54$  & $69.29$ \\ \hline
%fourier + ica??
\multicolumn{5}{c}{\textit{Deep Learning Methods}} \\ \hline
1D CNN raw  & $70000\times 2$         & $84.43$ & $67.68$ & $76.06$  \\ 
\textbf{2D CNN MTF} 
            & $48\times48\times 2$    & $84.26$   & $69.76$  & $77.01$   \\ 
\hline
%poner folded? yo diria que no .. ya que les va muy bien jaja y enrealidad hacen trampa al conocer el periodo.. (si se sabe el periodo mejor usar la metadata)
\end{tabular}
\end{table}
Table \ref{tab:class_results} compares different methods according to \textit{F1-score} metric.  %remarked in baseline section for the exoplanet detection problem by \textit{F1-score}. 
The classification for classic learning was made using the fully connected block of deep architectures (Table \ref{tab:model:arch}). This corresponds to 128 dense units with \textit{relu} activation function followed by the classification dense layer with one unit and \textit{sigmoid} activation function. 
Firstly, our method obtains a moderate F1-score macro averaged improvement of 0.95 over a 1D CNN, despite of the 1D CNN method uses a fairly sophisticated architecture with a large number of parameters (Table \ref{tab:model:arch}).
Drawing on 2D CNN model, our method extracts enough information to identify transit patterns on the MTF images.

Deep learning methods outperform the classic counterpart approach of F-PCA by $\sim 8\%$ and the hand-crafted features of feets showing a large gap ($\sim 33\%$). 
However, the specialized features for the problem, metadata, have the best performance for the task. 
Note that our method reaches the second best result, meaning that still are some improvements that could be done with the automatic techniques.

The detailed metrics per class show that the confirmed class is the most difficult to detect, meaning that the behavior of the exoplanets is not so clear in order to group and discriminate correctly all the patterns. 
The advantage from the metadata representation comes from this class, showing large improvement against all the other methods. 
The false positive class (non-exoplanets) appears easier to detect, based on the \textit{F1-score} higher than 80\% for all the methods. 
%While the false positives class (non-exoplanets) appears more easier to detect based on the probably clear patterns of the binary system or noise properties.

\begin{table}[!t]
\caption{Time comparison in seconds for different learning techniques. Values in parenthesis on Training column stands for the number of epochs to train the models. $\diamond$ This value is based on the information of \citep{fanelli2011kepler}, this is 4 days. $\star$ The feets representation time is measured over a subset of 2500 objects (30\% of the data).}
\label{tab:time_results}
\begin{tabular}{c|ccc|c} \hline
\textbf{Method} & \textbf{Representation} & \textbf{Training} & \textbf{Predict} & \textbf{Total} \\ \hline 
\multicolumn{5}{c}{\textit{Specialized hand-crafted features + Classic Learning Methods}} \\ \hline
Metadata   & $5760^\diamond$ & $37.8$ (200)  & $0.04$ & - \\ %\hline
feets      & $5912500^{\star}$   & $42.2$ (200)  & $0.07$ & $>100000$ mins \\ \hline %($2365$ x D)
\multicolumn{5}{c}{{\textit{Feature extraction + Classic Learning Methods}}} \\ \hline
F-PCA          & $166.6$  & $40.4$ (200) & $0.06$ & $\sim$ 4 mins \\ \hline  %\hline %$96.6$ (Fou) $0.012$ x D , $70$ (PCA
\multicolumn{5}{c}{{\textit{Deep Learning Methods}}} \\ \hline
1D CNN raw             & $0$ & $21500$ (50)  & $288.33$ & $\sim$ 360 mins  \\ 
\textbf{2D CNN MTF} 
                       & $145$ & $1040$ (200) & $4.72$ & $\sim$ 20 mins \\ %($0.018$ x D)
\hline
\end{tabular}
\end{table}
Table \ref{tab:time_results} shows the execution time by phase: i) the representation of the method, ii) training of the model and iii) predict; The total time is presented too.
Our method takes 20 minutes in average for running the complete process.  This includes the generation of all the MTF images for our dataset, training the 2D CNN model and predict on the test set partition.
This value is reasonable good compared to more specialized methods, such as the use of metadata (more than 4 days\footnote{This value is based on the information of \citep{fanelli2011kepler}} ) or feets features (more than 2 months).
%%%%%% opcion 1
%The execution time of our method becomes close but not faster than the classic method of F-PCA, but since our proposal managed to exceed this method in terms of macro averaged \textit{F1-score} is a reasonable trade-off.
%%% opcion 2
Despite the fact that the execution time of our method does not reach faster results than F-PCA, our method managed to outperform classic methods in terms of macro averaged \textit{F1-score}.

Finally, comparing our method against other deep learning techniques, we can see that our method has a shorter execution time than 1D CNN which uses the raw light curve. Despite 1D CNN method does not need a representation phase, this model has a quite high temporary cost on training and prediction steps due to the more complex processing. Our method focus on the representation but reduce time on training and prediction phases (a simpler model)
That is, using a 2D CNN model we can extract important information in a faster way, 18 times shorter: from 6 hours to just 20 minutes as total execution time.

\section{Interpreting the MTF Image Content}
\begin{figure}[t!]
\centering
    \includegraphics[width=.9\linewidth]{imgs/MTF_LC.png}
\caption{Examples of different types of behavior on MTF of Kepler mission. Only the confirmed objects (exoplanets) are shown here ($32\times 32$ matrix).}
\label{ex:mtf}
\end{figure}
The proposed MTF matrix is useful to illustrate the \textit{global} behavior of all the uneven measurements of light curves in a simpler way, specially if the number of measurements are more than 10 thousand values (as in our case).
The exoplanet transit problem allows us to interpret the light curve transitions channel. For example, the minor state corresponds to the moment when the planet is in front of its host star and blocks the maximum light. While the center state, on positive-negative symmetric matrix, means that the light of the star is measured cleanly (with no eclipsing objects or stellar noise).
 
Figure \ref{ex:mtf} shows different patterns in the transition channel of the MTF images. The objects correspond to confirmed exoplanets light curves of the dataset.
Here, is possible to identify two main types of behavior: diagonal and vertical. Within the diagonal patterns there are two types: ellipse and line, where the latter corresponds to an ellipse with maximum eccentricity ($e=\infty$). 

% + eccentricco + lineal  => lento proceso (periodo alto)
% - eccentrico + ovalado  => rapido proceso (periodo bajo)
The results allow us to discuss the following points.
The diagonal or ellipse patterns are present on those objects whose light curve accounts for clean measurements where eclipses are easily identify. 
Specifically, the level of eccentricity of transitions in MTF is related to the period of the transit object observed on the light curve. 
That is, curves with shorter period (faster transit) show more circular patterns (\textit{smaller eccentricity} on the ellipse) since they have more abrupt transitions between two contiguous points of the time series. 
Instead, a longer period (slower transit) indicates smooth transitions between the states, giving to more diagonal patterns (\textit{high eccentricity}) as observed in Figure \ref{ex:mtf} and Figure \ref{fig:mtf_ex}.
In summary, more eccentricity on the MTF behavior indicates slower orbiting planets.

\begin{figure}[t!]
    \centering
    \subfloat{{\includegraphics[width=0.45\textwidth]{imgs/Tiny_MTF1.png} }}%
    \qquad
    \subfloat{{\includegraphics[width=0.45\textwidth]{imgs/Tiny_MTF2.png} }}%
    \caption{Six examples of transition patterns observed on Kepler light curves. In the left is the measurement channel, transitions represented as probabilities (normalized). In the center is the time channel, from the temporal information of the light curve. On the right there is a binarized representation of the measurement channel, 1 if there is a transition between states.}%
    \label{fig:mtf_ex}
\end{figure}
On the other hand, vertical patterns are associated with objects whose light curve presents extremely diffuse transitions. These curves are generally accompanied by a high rate of measurement errors so transitions between states occur \textit{randomly} with the highest concentration at equilibrium states (central zone of the MTF). This phenomenon can be observed in the binary transition images (Figure \ref{fig:mtf_ex}) where there are transitions in almost all the states of the generated MTF. However, when the transitions counts are normalized, vertical patterns are observed.

Note that there are larger behavior patterns that cover the entire range of states. That is, there are measurements near the maximum or minimum value ($1 $ and $- 1$). 
Thus, larger patterns denote a significant number of transitions across the complete spectrum, while shallow patterns denote that the highest concentration of transitions occurs in central states with no focus on extreme values.


\begin{figure}[t!]
\includegraphics[width=.9\linewidth]{imgs/MTF_Time.png}
\caption{Example of different types of behavior on the time transitions MTF on Kepler mission. Only the confirmed objects (exoplanets) are shown here. $32\times 32$ matrix}
\label{ex:mtf_time}
\end{figure}
Following the analysis for the time channel on the MTF image representation, Figure \ref{ex:mtf_time} shows different observed behavior from confirmed exoplanets on the Kepler mission.
It can be seen that there are some continuous time transitions (exactly diagonal) and others with more discontinuous patterns (diagonal with cuts).
This last one occurs when the light curve transitions have a high delta time that does not meet the specified maximum delta $T_d$, this mean that some sampling rates are greater than the specified by Kepler (half an hour).
The discontinuous patterns are expected because the light curve could have irregularities based on the unevenly-sampled measurements of Kepler dataset, i.e. astronomical phenomenon that obstruct the observation.


\subsection{Model Prediction Analysis}
\begin{figure}[!t]
    \centering
\begin{tabular}{c}
    Confirmed (Exoplanet) predictions (with high confidence) \\
    \includegraphics[width=0.95\textwidth]{imgs/MTF_confE_data_v2.png}  \\
    False Positive (Non-Exoplanet) predictions (with high confidence) \\
     \includegraphics[width=0.95\textwidth]{imgs/MTF_confNE_data_v2.png} \\
    Difficult objects predictions (very low confidence) \\
    \includegraphics[width=0.95\textwidth]{imgs/MTF_diff_data_v2.png}
\end{tabular}
\caption{Examples on different types of predictions of the KOI objects (MTF).}
\label{fig:mtf_pred_ex}
\end{figure} %cmap es RdPu, sin limite en lognorm (el valor maximo y mas oscuro, no es 1)
Figure \ref{fig:mtf_pred_ex} shows examples of our representation for the KOI objects (MTF) that the 2D CNN model predicts with high reliability. We use the model probability predictions, $p(y=\texttt{confirmed})$, to understand the shape of objects with probability close to 1 (most likely confirmed) and close to 0 (most likely false positive). 

As previously stated, the standard well defined transits (without noise or random behavior) are objects with bottom right ellipse patterns, where the objects predicted as confirmed have more thick pattern than the objects predicted as false positives.
There also seems to be a second ellipse on the false positive objects, meaning perhaps a binary system (two stars eclipsing each other with two statistical different transits). 
These two intrinsic characteristic could be the factors that our model consider in order to classify as a certain class: analyze the type of pattern into the bottom diagonal (\textit{it is clearly only one ellipse?}) and how thick are the patterns.

In addition, we show the most difficult objects to classify, where the model is not sure of the class label (probability close to 0.5).
On this case it can be seen a big spot on the center with no clear inclination to the bottom right. These patterns are the vertical ones commented earlier which we associate to random transitions. 
This clarifies the difficulties of classify these types of objects despite the fact that some of them have an orbiting exoplanet. Therefore, there does not appear to be a clear difference between objects with or without exoplanets.

\section{Conclusions}
\label{concl}

In this work we propose to use variational (stochastic) autoencoder models to learn quality deep representation in the problem of transit modeling. 
We focus on adapting the variational autoencoders to properly handle unevenly-sampled light curves (time series) making improvements to their deterministic counterparts.
We presented a new model that includes the re-scaling pre-processing of time series into the model (as a end-to-end architecture), which leads to improvements on different evaluation schemes.

The evaluation on the learned representation shows that the variational proposed models have a higher quality. This mean that the representation is: i) more \textit{informative}, i.e. more independent features, so more information could be stored, ii) more \textit{useful}, i.e. more effective for the classification of exoplanets, and iii) more \textit{robust}, i.e. learn a noiseless light curve reconstruction.
By adding the re-scaling into the model, this three effects increase. For example, the S-VRAE$_t$ model ends up being almost as informative as the optimal PCA, at the same time that produces a denoising effect similar to a Mandel Agol fitted model.

Future work includes an extension of the re-scaled model that defines a learnable function built over the raw input measurements, i.e. $s^{(i)}= f(x^{(i)})$, and performing a sensitivity analysis on the architecture parameters. Also, we believe that interpreting these informative, useful and robust features by mapping each dimension to an astronomical concept, could enrich the current knowledge about exoplanets.



\section*{Acknowledgments}
This research was possible due to the funding of \textit{Programa de Iniciaci\'on Cient\'ifica} PIIC-DGIP of Universidad T\'ecnica Feder\'ico Santa Mar\'ia, ANID-Basal Project FB0008 (AC3E) and ANID PIA/APOYO AFB180002 (CCTVal).

\bibliography{mybibfile}

\end{document}