\section{Conclusion}

We propose an image representation based on Markov Transition Fields for unevenly-sampled light curves than can be combined with deep learning models. 
We apply this technique to the exoplanet detection problem, harnessing the power of 2D convolutional neural network as simpler alternative to deep learning methods than processing the raw representation of light curves.

The results over the KOI dataset of the Kepler mission takes around 20 minutes to process. Also, our method is faster to execute and lighter in terms of memory consumption, yet it offers a competitive performance of 77.01\% in the \textit{F1-score} (macro averaged) with respect to similar techniques. 

%We also show that the proposed method could be more effective for the learning task.

%The main advantage of our proposal is the execution time. This will help experts to generate early predictions of exoplanets, without the need to wait days or hours for another algorithm to generate a similar prediction.

Based on the visual interpretation of the MTF images content (global transition behavior), we plan to inspect in detail the least and most likely semi-continuous transitions in order to produce new knowledge from the specific local patterns of the time series. In addition, we have the intention to improve the sparse matrix representation investigating the use of kernel-based estimation in the continuous space.


