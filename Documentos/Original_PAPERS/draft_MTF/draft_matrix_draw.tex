%\documentclass[margin=0.5cm]{standalone}
%\usepackage{tikz}
%\usetikzlibrary{matrix,decorations.pathreplacing,calc,fit}

\pgfkeys{tikz/mymatrixenv/.style={decoration=brace,every left delimiter/.style={xshift=4.7pt},every right delimiter/.style={xshift=-4.7pt}}}
\pgfkeys{tikz/mymatrix/.style={matrix of math nodes, nodes in empty cells, left delimiter=[,right delimiter={]},inner sep=2pt,column sep=1em,row sep=0.5em,nodes={inner sep=0pt}}}
\pgfkeys{tikz/mymatrixbrace/.style={decorate,thick}}

% The hack required for foreach loops in fit. Code from https://tex.stackexchange.com/questions/4751/fitting-a-list-of-points-with-tikz-and-its-foreach?noredirect=1&lq=1
\makeatletter
\def\tikz@lib@fit@scan{%
  \pgfutil@ifnextchar\pgf@stop{\pgfutil@gobble}{%
    \pgfutil@ifnextchar\foreach{\tikz@lib@fit@scan@handle@foreach}{%
      \tikz@scan@one@point\tikz@lib@fit@scan@handle}}}
\def\tikz@lib@fit@scan@handle@foreach\foreach#1in#2#3{%
  \foreach #1 in {#2}
  {\tikz@scan@one@point\tikz@lib@fit@scan@handle@foreach@#3}
  \tikz@lib@fit@scan}
\def\tikz@lib@fit@scan@handle@foreach@#1{%
  \iftikz@shapeborder
    \tikz@lib@fit@adjust{%
      \pgfpointanchor{\tikz@shapeborder@name}{west}}%
    \tikz@lib@fit@adjust{%
      \pgfpointanchor{\tikz@shapeborder@name}{east}}%
    \tikz@lib@fit@adjust{%
      \pgfpointanchor{\tikz@shapeborder@name}{north}}%
    \tikz@lib@fit@adjust{%
      \pgfpointanchor{\tikz@shapeborder@name}{south}}%
  \else
    \tikz@lib@fit@adjust{#1}%
  \fi
  \global\pgf@xa=\pgf@xa
  \global\pgf@ya=\pgf@ya
  \global\pgf@xb=\pgf@xb
  \global\pgf@yb=\pgf@yb}
\makeatletter

%\begin{document}

\begin{figure}[!t]
    \centering
    %\includegraphics{}
    
\begin{tikzpicture}[baseline=0cm,mymatrixenv]
    \matrix [mymatrix,outer ysep=0.7pt,inner sep=4pt,row sep=1em] (m)  
    {
    m_{1,1}  &  m_{1,2} &  m_{1,3} & m_{1,4} & m_{1,5} &  m_{1,6} & \dots & m_{1,n}  \\
    m_{2,1}  &  m_{2,2} &  m_{2,3} & m_{2,4} & m_{2,5} &  m_{2,6} & \dots & m_{2,n}  \\
    m_{3,1}  &  m_{3,2} &  m_{3,3} & m_{3,4} & m_{3,5} &  m_{3,6} & \dots & m_{3,n}  \\
    m_{4,1}  & m_{4,2} & m_{4,3} & m_{4,4} & m_{4,5} & m_{4,6} & \dots & m_{4,n} \\
   m_{5,1}   & m_{5,2} & m_{5,3} & m_{5,4} & m_{5,5} & m_{5,6}  & \dots & m_{5,n} \\
    m_{6,1}  & m_{6,2} & m_{6,3} & m_{6,4} & m_{6,5}  & m_{6,6} & \dots & m_{6,n}\\
    \vdots   & \vdots  & \vdots  & \vdots  & \vdots  & \vdots  & \ddots & \vdots \\
    m_{n,1}  & m_{n,2} & m_{n,3} & m_{n,4} & m_{n,5} & m_{n,6} & \dots & m_{n,n} \\
    };

% Colours
\definecolor{brightpurple}{HTML}{C151EF}

\node [fit= \foreach \X in {1,...,3}{(m-\X-1)}
            \foreach \X in {1,...,3}{(m-\X-2)}
            \foreach \X in {1,...,3}{(m-\X-3)}]
            [draw=green, thick,inner sep=2.6pt] (fit-a) {};

\node [fit= \foreach \X in {4,...,6}{(m-\X-4)}
            \foreach \X in {4,...,6}{(m-\X-5)}
            \foreach \X in {4,...,6}{(m-\X-6)}]
            [draw=cyan, thick,inner sep=2.6pt] (fit-b) {};

\node [fit= \foreach \X in {2,...,4}{(m-\X-2)}
            \foreach \X in {2,...,4}{(m-\X-3)}
            \foreach \X in {2,...,4}{(m-\X-4)}]
            [draw=orange, thick,inner sep=2.6pt] (fit-c) {};
            
\node [fit= \foreach \X in {3,...,5}{(m-\X-3)}
            \foreach \X in {3,...,5}{(m-\X-4)}
            \foreach \X in {3,...,5}{(m-\X-5)}]
            [draw=purple, thick,inner sep=2.6pt] (fit-d) {};
            
% FINDING VERTICAL MIDPOINT
 \node [fit= \foreach \X in {1,...,3}{
            (m-\X-1)}] (fit-one)  {}; 
 \node [fit= \foreach \X in {4,...,6}{
            (m-\X-6)}] (fit-two)  {}; 
\path (fit-one.south) -- (fit-two.north) coordinate[midway] (X);

% FINDING HORIZONTAL MIDPOINT
 \node [fit= \foreach \X in {1,...,3}{
            (m-1-\X)}] (fit-one)  {}; 
 \node [fit= \foreach \X in {4,...,6}{
            (m-6-\X)}] (fit-two)  {}; 
\path (fit-one.east) -- (fit-two.west) coordinate[midway] (Y);

\newcommand\mymatrixbraceoffseth{0.3em}
\newcommand\mymatrixbraceoffsetv{0.3em}

% LHS BRACES
\draw [mymatrixbrace] ($(m.north west)!(fit-a.south)!(m.south west)-(\mymatrixbraceoffseth,0)$)   -- node[left=3pt] {$K_1$}  ($(m.north west)!(fit-a.north)!(m.south west)-(\mymatrixbraceoffseth,0)$);
\draw [mymatrixbrace] ($(m.north west)!(fit-b.south)!(m.south west)-(\mymatrixbraceoffseth,0)$) -- node[left=3pt] {$K_2$} ($(m.north west)!(fit-b.north)!(m.south west)-(\mymatrixbraceoffseth,0)$);

% TOP BRACES       
\draw[mymatrixbrace] ($(m.north west)!([xshift=0.05cm]Y)!(m.north east)+(0,\mymatrixbraceoffsetv)$) -- node[above=3pt] {$K_2''$}  ($(m.north west)!(fit-b.east)!(m.north east)+(0,\mymatrixbraceoffsetv)$);
\draw[mymatrixbrace] ($(m.north west)!(fit-a.west)!(m.north east)+(0,\mymatrixbraceoffsetv)$)-- node[above=3pt] {$K_1'$}   ($(m.north west)!([xshift=-0.05cm]Y)!(m.north east)+(0,\mymatrixbraceoffsetv)$);

\end{tikzpicture}
\caption{Illustration of a Markov Transition Field (MTF) matrix. Each entry into the matrix ($m_{i, j}$) corresponds to a defined state. Each box corresponds to a feature map for a $3 \times 3$ kernel size.}
\label{fig:matrix_MTF}
\end{figure}


%\end{document}